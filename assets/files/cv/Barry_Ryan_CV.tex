\documentclass[letterpaper,10pt]{article}
\usepackage{enumitem}
\usepackage{geometry}
\usepackage[none]{hyphenat}
\geometry{
    top=0.25in,      % Top margin
    bottom=0.25in, % Bottom margin
    left=0.5in,   % Left margin
    right=0.5in    % Right margin
}
\usepackage{hyperref}
\hypersetup{
    colorlinks=true, % false: boxed links; true: colored links
    linkcolor=black,  % color of internal links (change box color with linkbordercolor)
    citecolor=blue,  % color of links to bibliography
    filecolor=magenta, % color of file links
    urlcolor=black,   % color of external links
    pdfborder={0 0 0}, % remove borders around links
    pdfnewwindow=true, % links open in new window
    pdfstartview={XYZ null null 1.00} % adjust PDF view
}


% Custom command for section headings formatting
\newcommand{\heading}[1]{%
    \vspace{-5mm} % Adjust the negative vertical space to fit closely
    \section*{#1}%
    \vspace{-3mm}%
    \noindent\hrule height 0.5pt % No space under the heading
    \vspace{4mm}%
}

% Custom command for experience block
% Usage: \experience{Title}{Company}{Location}{Date}{Bullet points}
\newcommand{\experience}[5]{%
    % \vspace{-1mm}%
    \noindent\textbf{#1}%
    % Check if #2 (Company) is provided
    \ifx&#2&
    % If #2 is empty, skip the comma and company
    \else
        , \textit{#2}%
    \fi
    % Check if #3 (Location) is provided
    \ifx&#3&
    % If #3 is empty, skip the comma and location
    \else
        , #3%
    \fi
    \hfill \textit{#4} \\
    \vspace{-6.8mm}%
    \begin{itemize}[itemsep=-5pt]
        \setlength{\itemindent}{0em}
        #5
    \end{itemize}
    \vspace{1mm}
}
\newcommand{\blankexperience}[5]{%
    \vspace{-1mm}%
    \noindent\textbf{#1}%
    % Check if #2 (Company) is provided
    \ifx&#2&
    % If #2 is empty, skip the comma and company
    \else
        , \textit{#2}%
    \fi
    % Check if #3 (Location) is provided
    \ifx&#3&
    % If #3 is empty, skip the comma and location
    \else
        , #3%
    \fi
    \hfill \textit{#4} \\
    \vspace{-1.5mm}%
}

\begin{document}

% Personal Information
\begin{center}
    \vspace{-5mm} 
    \textbf{\huge Ryan Barry} \\
    \vspace{1mm}
    \href{http://ryanbarry.me}{ryanbarry.me}
    \hspace{0.2em} \vline \hspace{0.2em}
    \vspace{1mm}
    \href{http://github.com/ryan-barry-99}{github.com/ryan-barry-99}
    \hspace{0.2em} \vline \hspace{0.2em}
    \href{mailto:ryanbarry.engineering@gmail.com}{ryanbarry.engineering@gmail.com}
    \hspace{0.2em} \vline \hspace{0.2em}
    (315) 596-0754
\end{center}

% Professional Experience
\vspace{-3mm} % Adjust the negative vertical space to fit closely

% Education 
\vspace{-0.5em}
\heading{Education}
\vspace{-1mm}%
\noindent\textbf{Rochester Institute of Technology}, Rochester, NY \\
Master of Science in Electrical Engineering \hfill Cumulative GPA: 3.92 \\
\textit{Specialization in Robotics and AI/ML} \\ \\
\noindent\textbf{Rochester Institute of Technology}, Rochester, NY \\
Bachelor of Science in Electrical Engineering, \textit{Summa Cum Laude} \hfill Cumulative GPA: 3.86 \\ 
\textbf{Tau Beta Pi Engineering Honor Society} \\
% \textbf{Applicable Courses:} Robotic-Systems, Principles-of-Robotics, Advanced-Robotics, Intro-to-Artificial-Intelligence, AI-Explorations, Biorobotics/Machine-Learning, Deep-Learning, Robot-Perception



% Skills
% \heading{Technical Skills}
% \vspace{-1mm}
% \noindent
% \textbf{Programming Languages:} Python, C/C++, MATLAB, Assembly, PLC Ladder Logic \\
% \textbf{Machine Learning / AI:} PyTorch, TensorFlow, Keras, Scikit-Learn, NumPy, Pandas, Jupyter, CUDA \\
% \textbf{Computer Vision \& Robotics:} OpenCV, Object Detection \& Tracking, ROS \\
% \textbf{Hardware \& Embedded Systems:} Microcontrollers, PCB Design, Motor/Sensor Integration, SMT/TH Soldering \\
% \textbf{CAD \& Mechanical Tools:} SOLIDWORKS, Creo, Inventor, AutoCAD, 3D Printing \\
% \textbf{Simulation / Analysis:} MATLAB, Spice, ROS, Hardware-in-the-Loop Environments \\
% \textbf{Workflow \& Tooling:} Git, Bash, Docker, LaTeX \\

\heading{Technical Skills}
\vspace{-1mm}%
\noindent\textbf{Languages:} Assembly, C/C++, PLC Ladder Logic, Python \\
\textbf{Libraries \& Tools:} CUDA, Git, Jupyter, Keras, NumPy, OpenCV, Pandas, PyTorch, ROS, Scikit-Learn, TensorFlow \\
\textbf{Software:} Altium Designer, AutoCAD, Creo, Inventor, LTSpice, MATLAB, SOLIDWORKS \\
\textbf{Hardware:} Microcontrollers, Motor/Sensor Control, PCB Design, SMT \& TH Soldering, 3D Printing \\


\heading{Professional Experience and Research}

\experience
    {Robotics Research Engineer II}
    {Robotics and Automation Design Lab}
    {Bryan, TX}
    {March 2025 – Present}
    {
        \item Designed fault-tolerant robotic manipulators for microgravity environments, emphasizing modularity and resilience to extreme temperature shifts and radiation.
        \item Designed custom PCBs in Altium for testing and spaceflight applications.
        \item Developed firmware and software in ROS, C++, and Python for actuator control and status monitoring.
        \item Built real-time data acquisition tools in C++ and Python for diagnostic logging, fault response validation, and performance analysis.
    }

\experience
    {Researcher}
    {RIT Adaptive Human-Robot Teaming Lab}
    {Rochester, NY}
    {August 2023 – May 2024}
    {
        \item  Developed a custom reinforcement learning (RL) environment in ROS and Gazebo for terrain-aware velocity control of a quadruped robot in a physics-based simulator.
        \item  Built a custom PPO reinforcement learner in PyTorch with LSTM-based policy and value networks; integrated the full pipeline into ROS for trajectory planning.
    }
    
\experience
    {Robotics Graduate Teaching Assistant}
    {Rochester Institute of Technology}
    {Rochester, NY}
    {August 2023 – May 2024}
    {
        \item  Facilitated student learning of high-level robotics concepts and ROS through lab work and research projects.
    }

\experience
    {\href{https://ryanbarry.me/projects/robotics/rover/}{Software Technical Lead}}
    {\href{https://ryanbarry.me/projects/robotics/rover/}{RIT University Rover Challenge Team}}
    {Rochester, NY}
    {June 2023 – May 2024}
    {
        \item Led architecture and full-stack ROS software development for an autonomous planetary rover; managed subsystem integration and testing across a 5-person team.
        \item Integrated a Python ROS application with embedded C++ microcontrollers via CAN to control distributed subsystems.
        \item Led cross-functional design reviews to ensure electromechanical subsystems aligned with software architecture requirements.
        % \item \href{https://ryanbarry.me/projects/robotics/rover/}{More at https://ryanbarry.me/projects/robotics/rover/}
    }

\experience
    {Electrical Engineer}
    {RIT Electric Vehicle Team}
    {Rochester, NY}
    {August 2021 – May 2024}
    {
        \item Designed a CAN interface board for a BeagleBone Black to communicate with the network of custom electric motorcycles.
        \item Led a team of undergraduate electrical engineers to develop a CAN based IMU board in Altium. 
        \item Engineered and documented gate driver + control architecture for a 3-phase BLDC motor from first principles, enabling powertrain control for brushless coolant pump; mentored junior team members in implementing the full motor controller in Altium as a foundation for future high-power traction systems.

    }

\experience
    {\href{https://ryanbarry.me/projects/ai-ml/sonar-data-pipeline/}{Machine Learning R\&D Intern}}
    {Penn State ARL}
    {University Park, PA}
    {May 2023 – August 2023}
    {
        \item Built synthetic data generation pipeline for active acoustic ML models in unmanned undersea vehicles (UUVs), reducing dependency on scarce labeled datasets.
        \item Designed Python application for scenario development and interface with UUV simulation software.
        \item Developed an acoustic range and angle of arrival regression model to support transfer learning hypothesis.
        % \item \href{https://ryanbarry.me/projects/ai-ml/sonar-data-pipeline/}{More at https://ryanbarry.me/projects/ai-ml/sonar-data-pipeline/}
    }

\experience
    {Product Engineering Co-op}
    {The Raymond Corporation}
    {Greene, NY}
    {January 2022 – July 2022}
    {
        \item Programmed PLC-based test fixture with touchscreen UI for reliability testing of forklift control cables.
        
        \item Supported CAN system emulation for motor controller validation.
        % \item Programmed a PLC based robotic test fixture with a touch LCD to stress test forklift control handle cables.

        % \item Researched and documented the adaptation of a software application to emulate CAN functionality for motor controllers on PCBs used in forklift test fixtures.

    }

\experience
    {Electrical Engineering Intern}
    {Davis Standard LLC}
    {Fulton, NY}
    {May 2021 – August 2021}
    {
        \item Revised schematics and drawings in AutoCAD; developed early prototype for 3D printer-inspired polymer extruder.
        
        % \item Revised wiring schematic and electrical panel drawings in AutoCAD.

        % \item Researched and developed proof of concept prototype for an industrial polymer extruder modeled after 3D Printers to be used for automated splice line placement along material necking border.
    }
    


% Projects
\vspace{-0.5em}
\heading{Projects}
% \vspace{1em}
{\Large \noindent\textbf{Highlighted Projects}} \\
% \vspace{-1em}

% \experience
%     {Custom Hybrid Electric Wheelchair Attachment}
%     {}
%     {}
%     {September 2024 – Present}
%     {
%         \item Designing a non-invasive hub actuation system for low drag, hybrid manual and powered control of manual wheelchairs.
%     }
    
\experience
    {\href{https://ryanbarry.me/projects/robotics/kinematics/}{Open-Source Universal Kinematic Libraries for Generic Robots}}
    {}
    {}
    {September 2023 – Present}
    {
        \item Developed modular C++ and Python libraries for forward and inverse kinematics of both serial-link manipulators and fixed-wheeled mobile robots.
        \item Enabled dynamic configuration from DH parameters or wheel layouts to support arbitrary robot topologies without rewriting core math.
        \item Implemented forward velocity kinematics and inverse kinematics for mobile platforms using wheel geometry
        \item Solved numerical inverse kinematics using Jacobian pseudo-inverse methods with tolerance-based convergence on joint angles from target end-effector pose for arbitrary robot configurations.
        % \item \href{https://ryanbarry.me/projects/robotics/kinematics/}{More at https://ryanbarry.me/projects/robotics/kinematics/}
        % \item \href{https://github.com/ryan-barry-99/mobile_robot_kinematics_cpp}{Developed Python and C++ libraries for forward and inverse kinematic calculations of any fixed-wheeled mobile robot.}
        
        % \item \href{https://github.com/ryan-barry-99/arm_robot_kinematics_cpp/tree/main}{Developed Python and C++ libraries for forward and inverse kinematic calculations of any arm robot configuration.}
    }
    
\experience
    {Multi-Agent Reinforcement Learning for Pacman Capture the Flag}
    {}
    {}
    {November 2023 – December 2023}
    {
        \item Developed a dual-agent Q-learning system with handcrafted reward shaping and dense feature vectors to coordinate offensive and defensive roles in a 2v2 Capture the Flag game.
        \item Built a shared memory mechanism for real-time inter-agent communication, enabling emergent ambush and retreat behaviors.
        \item Implemented all learning and inference logic from scratch in a single script without external ML libraries; stored and updated network weights in script-local dictionaries.
        \item Trained agents via self-play and curriculum learning against rotating baseline opponents across randomized maps.
        \item Reached tournament finals; a late-stage regression bug in retreat logic impacted final match performance.
   % \item \href{https://github.com/ryan-barry-99/AI_Explorations_Pacman_Capture_the_Flag}{SARSA-trained agents with shared memory and reward shaping; reached tournament finals.}
    }


\experience
    {\href{https://ryanbarry.me/projects/robotics/air-hockey-robot/}{Air Hockey Robot}}
    {}
    {}
    {October 2023 – December 2023}
    {
        \item Engineered a full-stack robotic system with a 3-DOF planar arm, overhead camera, and real-time closed-loop inference.
        \item Collected and labeled 26,627 training images to train a YOLOv8 model from scratch for puck and keypoint detection; validated with PR curves and batch predictions.
        \item Predicted puck trajectories using a physics model and an LSTM trained on bidirectional crossing sequences via data augmentation; tuned for end-effector interception within a 4-inch spatial margin.
        \item Deployed a safety-bounded inverse kinematics controller with joint angle lookup at 57 FPS inference throughput, bottlenecked only by the 60 FPS camera.
    }



    
\experience
    {\href{https://ryanbarry.me/projects/robotics/omnidirectional-robot/}{Full-Stack Robot for Real-Time Object Interception}}
    {}
    {}
    {February 2023 – April 2023}
    {
        % \item Designed and programmed an omnidirectional robot to serve as a modular research platform.

        \item Designed and programmed a holonomic robot with real-time YOLOv8-based object tracking and trajectory alignment.
        \item Designed custom chassis design with 3D-printable omni wheels in SOLIDWORKS.
        \item Trained a custom YOLOv8 model to detect balloons and calculated their 3D velocity with a stereo camera using weighted difference method over a buffer of frames.
        \item Developed a ROS network for kinematic motion control and sensor integration.
        % \item More at \href{https://ryanbarry.me/projects/robotics/omnidirectional-robot/}{https://ryanbarry.me/projects/robotics/omnidirectional-robot/}
    }
    
% \experience
%     {\href{https://ryanbarry.me/projects/electrical-engineering/underwater-robot/}{Underwater Robot Motherboard PCB}}
%     {}
%     {}
%     {November 2022 – April 2023}
%     {
%         \item Integrated a 7-sensor perception suite with power distribution and propulsion control on a custom PCB.
%         \item Sensor peripherals include: 2 cameras, dissolved oxygen sensor, IMU, pH probe, salinity sensor, thermistor.
%         \item Ethernet controller for high speed data transfer to external laptop.
%         \item Power regulation for 11V, 5V, and 3.3V rails from a 12V nominal battery.
%         \item High-current H-Bridge for brushed DC motor control of a propeller. Additionally a servo header for rudder control.
%         \item Designed using Altium Designer and hand soldered SMT components.
%         \item More at \href{https://ryanbarry.me/projects/electrical-engineering/underwater-robot/}{https://ryanbarry.me/projects/electrical-engineering/underwater-robot/}
        
%         % \item \href{https://ryanbarry.me/projects/electrical-engineering/underwater-robot/}{Designed and manufactured a motherboard PCB capable of managing power distribution, motor control, servo control, interfacing with two cameras, and integrating five sensors.}
%     }
    
% \experience
%     {Collaborative Robot Ball-and-Cup Game}
%     {}
%     {}
%     {February 2023 – April 2023}
%     {
%         \item Built a ROS-based interactive game for the Baxter robot using YOLOv6 and DeepSORT with occlusion-aware ID tracking; designed a custom parallel-jaw-compatible gripper for robust cup manipulation.
%     }
    


% % \experience
% %     {Multimodal Fusion Architecture for IMDb Rating Estimation}
% %     {}
% %     {}
% %     {April 2023}
% %     {
% %         \item Built a PyTorch model combining BERT and a CNN for multimodal regression on IMDb ratings from movie posters and text descriptions.
% %         \item Achieved 81.3\% validation accuracy for predictions within 0.3 rating margin on a real-world dataset.
% %     }

% \experience
%     {Nerf Mobile Robot with Autonomous Targeting}
%     {}
%     {}
%     {November 2021 – December 2021}
%     {
%         \item Built a mobile robot with autonomous tracking and firing at a moving shield using a modified Nerf blaster.
%         \item Utilized bounding box color tracking for target localization and aiming.
%         \item Controlled trigger and flywheel with dedicated servo and logic-controlled high-side power switch.
%     }

% \vspace{1em}
% {\Large \noindent\textbf{Additional Projects}} \\
% \vspace{-1em}
% \begin{itemize}
%     \item In progress design of a non-invasive wheelchair attachment for low drag, hybrid manual and powered control.
%     \item \href{https://ryanbarry.me/projects/electrical-engineering/underwater-robot/}{Robot motherboard PCB with 7 sensor perception suite, power distribution, and propultion control.}
%     \item EMG signal live gesture classification pipeline using custom model trained from self-collected bio signals.
%     \item Interactive ball-and-cup game for Baxter robot using object detection and tracking with occlusion aware ID persistence.
%     \item Nerf mobile robot with autonomous targeting and firing system.
% \end{itemize}
{\Large \noindent\textbf{Additional Projects}} \\
\vspace{-1em}
\begin{itemize}
    \item \href{https://ryanbarry.me}{\textbf{Personal Portfolio Website \textit{(Ongoing)}:} Designed and developed a custom HTML/CSS website hosted on GitHub Pages to showcase engineering projects, complete with interactive documentation, videos, and embedded reports.}

    \item \textbf{Wheelchair Attachment Prototype \textit{(In Progress)}:} Prototyping a non-invasive, low-drag wheelchair add-on with hybrid manual and powered control.
    
    \item \href{https://ryanbarry.me/projects/electrical-engineering/underwater-robot/}{\textbf{Underwater Robot Motherboard PCB \textit{(Spring 2023)}:} Designed and integrated a PCB with a 7-sensor perception suite, power distribution, and propulsion control.}
    
    \item \textbf{EMG Gesture Classification Pipeline \textit{(Spring 2023)}:} Built a real-time gesture classifier using a custom-trained neural network on self-collected EMG biosignals.
    
    \item \textbf{Baxter Robot Interactive Game \textit{(Fall 2022)}:} Developed a ball-and-cup game using object tracking and occlusion-aware ID persistence.
    
    \item \textbf{Autonomous Nerf Sentry \textit{(Fall 2021)}:} Designed a mobile robot with onboard targeting, tracking, and automated firing system.
\end{itemize}
% \experience
%     {\href{https://ryanbarry.me/projects/electrical-engineering/underwater-robot/}{Underwater Robot Motherboard PCB}}
%     {}
%     {}
%     {November 2022 – April 2023}
%     {
%         \item Integrated a 7-sensor perception suite with power distribution and propulsion control on a custom PCB.
        
%         % \item \href{https://ryanbarry.me/projects/electrical-engineering/underwater-robot/}{Designed and manufactured a motherboard PCB capable of managing power distribution, motor control, servo control, interfacing with two cameras, and integrating five sensors.}
%     }

% \experience
%     {Collaborative Robot Interactive Ball-and-Cup Game}
%     {}
%     {}
%     {February 2023 – April 2023}
%     {
%         \item Built a ROS-based game for Baxter using YOLOv6 and DeepSORT with occlusion-aware ID tracking.
%     }

% \experience
%     {Nerf Mobile Robot with Autonomous Targeting}
%     {}
%     {}
%     {November 2021 – December 2021}
%     {
%         \item Built a mobile robot with autonomous tracking and firing at a moving shield using a modified Nerf blaster.
%     }

\end{document}
