\documentclass[letterpaper,10pt]{article}
\usepackage{enumitem}
\usepackage{geometry}
\geometry{
    top=0.25in,      % Top margin
    bottom=0.25in, % Bottom margin
    left=0.5in,   % Left margin
    right=0.5in    % Right margin
}
\usepackage{hyperref}
\hypersetup{
    colorlinks=true, % false: boxed links; true: colored links
    linkcolor=black,  % color of internal links (change box color with linkbordercolor)
    citecolor=blue,  % color of links to bibliography
    filecolor=magenta, % color of file links
    urlcolor=black,   % color of external links
    pdfborder={0 0 0}, % remove borders around links
    pdfnewwindow=true, % links open in new window
    pdfstartview={XYZ null null 1.00} % adjust PDF view
}


% Custom command for section headings formatting
\newcommand{\heading}[1]{%
    \vspace{-5mm} % Adjust the negative vertical space to fit closely
    \section*{#1}%
    \vspace{-3mm}%
    \noindent\hrule height 0.5pt % No space under the heading
    \vspace{4mm}%
}

% Custom command for experience block
% Usage: \experience{Title}{Company}{Location}{Date}{Bullet points}
\newcommand{\experience}[5]{%
    \vspace{-1mm}%
    \noindent\textbf{#1}%
    % Check if #2 (Company) is provided
    \ifx&#2&
    % If #2 is empty, skip the comma and company
    \else
        , \textit{#2}%
    \fi
    % Check if #3 (Location) is provided
    \ifx&#3&
    % If #3 is empty, skip the comma and location
    \else
        , #3%
    \fi
    \hfill \textit{#4} \\
    \vspace{-6.8mm}%
    \begin{itemize}[itemsep=-5pt]
        \setlength{\itemindent}{0em}
        #5
    \end{itemize}
}

\begin{document}

% Personal Information
\begin{center}
    \vspace{-5mm} 
    \textbf{\huge Ryan Barry} \\
    \vspace{1mm}
    \href{http://ryanbarry.me}{ryanbarry.me}
    \hspace{0.2em} \vline \hspace{0.2em}
    \href{mailto:ryanbarry.engineering@gmail.com}{ryanbarry.engineering@gmail.com}
    \hspace{0.2em} \vline \hspace{0.2em}
    (315) 596-0754
\end{center}

% Professional Experience
\vspace{-3mm} % Adjust the negative vertical space to fit closely


% Skills
\heading{Technical Skills}
\vspace{-1mm}%
\noindent\textbf{Languages:} Assembly, C/C++, MATLAB, PLC Ladder Logic, Python \\
\textbf{Frameworks:} CUDA, Git, Jupyter, Keras, NumPy, OpenCV, Pandas, PyTorch, ROS, Scikit-Learn, TensorFlow \\
\textbf{Software:} Altium Designer, AutoCAD, Creo, Inventor, LTSpice, MATLAB, SOLIDWORKS \\
\textbf{Hardware:} Microcontrollers, Motor/Sensor Control, PCB Design, SMT Soldering, TH Soldering, 3D Printing


\heading{Professional Experience and Research}

\experience
    {Robotics Research Engineer II}
    {Robotics and Automation Design Lab}
    {Bryan, TX}
    {March 2025 – Present}
    {
        \item Engineering multi-fault tolerant robotic manipulators for use in contracted space missions, designed for environmental robustness and modularity.
        \item Designed custom PCBs in Altium for testing and spaceflight applications.
        \item Developed firmware and software in ROS, C++, and Python for actuator control and status monitoring.
        \item Built real-time data acquisition tools in C++ and Python for diagnostic logging, fault response validation, and performance analysis.
    }

\experience
    {Researcher}
    {RIT Adaptive Human-Robot Teaming Lab}
    {Rochester, NY}
    {August 2023 – May 2024}
    {
        \item  Developed a custom reinforcement learning (RL) environment in ROS and Gazebo for terrain-aware velocity control of a quadruped robot in a physics-based simulator.
        \item  Implemented Proximal Policy Optimization (PPO) from scratch in PyTorch with LSTM-based policy and value networks; built a custom trajectory buffer and integrated the full RL pipeline into the ROS-based system.
    }
    
\experience
    {Robotics Graduate Teaching Assistant}
    {Rochester Institute of Technology}
    {Rochester, NY}
    {August 2023 – May 2024}
    {
        \item  Facilitated student learning of high-level robotics concepts and ROS through lab work and research projects.
    }

\experience
    {\href{https://github.com/ryan-barry-99/rovers}{Software Technical Lead}}
    {\href{https://github.com/ryan-barry-99/rovers}{RIT University Rover Challenge Team}}
    {Rochester, NY}
    {June 2023 – May 2024}
    {
        \item \href{https://github.com/ryan-barry-99/rovers}{Spearheaded software architecture development for autonomous and remote operation of a robotic rover.}

        \item \href{https://github.com/ryan-barry-99/rovers}{Directed a team of computer scientists to develop and test software for all subsystems of RIT’s rover.}

        \item \href{https://github.com/ryan-barry-99/rovers}{Integrated a Python-based ROS application with embedded C++ code for peripheral control via CAN.}
    }

\experience
    {Electrical Engineer}
    {RIT Electric Vehicle Team}
    {Rochester, NY}
    {August 2021 – May 2024}
    {
        \item Designed PCBs for custom electric motorcycles (IMU, CAN data handler, and 3 phase motor controller).

        \item Led teams and mentored underclassmen in PCB design using Altium, and in research and development principles.

    }

\experience
    {\href{https://ryanbarry.me/projects/ai-ml/sonar-data-pipeline/}{Machine Learning R\&D Intern}}
    {Penn State ARL}
    {University Park, PA}
    {May 2023 – August 2023}
    {
        \item Developed proof of concept synthetic data pipeline for active acoustic ML in unmanned undersea vehicles.

        \item Designed Python application for scenario development and interface with UUV simulation software.

        \item Developed an acoustic range and angle of arrival regression model to support transfer learning hypothesis.
    }

\experience
    {Product Engineering Co-op}
    {The Raymond Corporation}
    {Greene, NY}
    {January 2022 – July 2022}
    {
        \item Programmed PLC-based test fixture with touchscreen UI for reliability testing of forklift control cables.
        
        \item Supported CAN system emulation for motor controller validation.
        % \item Programmed a PLC based robotic test fixture with a touch LCD to stress test forklift control handle cables.

        % \item Researched and documented the adaptation of a software application to emulate CAN functionality for motor controllers on PCBs used in forklift test fixtures.

    }

\experience
    {Electrical Engineering Intern}
    {Davis Standard LLC}
    {Fulton, NY}
    {May 2021 – August 2021}
    {
        \item Revised schematics and drawings in AutoCAD; developed early prototype for 3D printer-inspired polymer extruder.
        
        % \item Revised wiring schematic and electrical panel drawings in AutoCAD.

        % \item Researched and developed proof of concept prototype for an industrial polymer extruder modeled after 3D Printers to be used for automated splice line placement along material necking border.
    }
    


% Projects
\vspace{-0.5em}
\heading{Projects}

% \experience
%     {Custom Hybrid Electric Wheelchair Attachment}
%     {}
%     {}
%     {September 2024 – Present}
%     {
%         \item Designing a non-invasive hub actuation system for low drag, hybrid manual and powered control of manual wheelchairs.
%     }
    
\experience
    {Open-Source Universal Kinematic Libraries for Generic Robots}
    {}
    {}
    {September 2023 – Present}
    {
        \item \href{https://ryanbarry.me/projects/robotics/kinematics/}{Developed Python and C++ libraries for forward/inverse kinematics of generic arm and fixed-wheeled mobile robots.}
        % \item \href{https://github.com/ryan-barry-99/mobile_robot_kinematics_cpp}{Developed Python and C++ libraries for forward and inverse kinematic calculations of any fixed-wheeled mobile robot.}
        
        % \item \href{https://github.com/ryan-barry-99/arm_robot_kinematics_cpp/tree/main}{Developed Python and C++ libraries for forward and inverse kinematic calculations of any arm robot configuration.}
    }
    
\experience
    {Air Hockey Robot}
    {}
    {}
    {October 2023 – December 2023}
    {
        \item \href{https://ryanbarry.me/projects/robotics/air-hockey-robot/}{Designed a 3-DOF air hockey robot using YOLOv8 for puck detection and LSTM-based trajectory prediction.}
    }


    
% \experience
%     {Multimodal Fusion Architecture for IMDb Rating Estimation}
%     {}
%     {}
%     {April 2023}
%     {
%         \item Developed a Convolutional Neural Network (CNN) and BERT fusion-based regression model for IMDb rating estimation from movie posters and related text. Achieved 81.3\% validation accuracy for predictions within 0.3 rating margin.
%     }


    
\experience
    {\href{https://github.com/ryan-barry-99/Trajectory-Matching-Omnidirectional-Robot/blob/main/Real_time_Trajectory_Matching_of_Moving_Objects.pdf}{Trajectory Matching Omnidirectional Mobile Robot}}
    {}
    {}
    {February 2023 – April 2023}
    {
        % \item Designed and programmed an omnidirectional robot to serve as a modular research platform.

        \item \href{https://github.com/ryan-barry-99/Trajectory-Matching-Omnidirectional-Robot/tree/main/}{Designed and programmed a holonomic robot with real-time YOLOv8-based object tracking and trajectory alignment.}
    }
    
\experience
    {\href{https://ryanbarry.me/projects/electrical-engineering/underwater-robot/}{Underwater Robot Motherboard PCB}}
    {}
    {}
    {November 2022 – April 2023}
    {
        \item \href{https://ryanbarry.me/projects/electrical-engineering/underwater-robot/}{Designed and soldered a PCB to integrate power distribution, propulsion and rudder actuation, and 7 onboard sensors.}
        
        % \item \href{https://ryanbarry.me/projects/electrical-engineering/underwater-robot/}{Designed and manufactured a motherboard PCB capable of managing power distribution, motor control, servo control, interfacing with two cameras, and integrating five sensors.}
    }
    
\experience
    {Collaborative Robot Ball-and-Cup Game}
    {}
    {}
    {February 2023 – April 2023}
    {
        \item \href{https://github.com/ryan-barry-99/DeepSORT-with-OAK-D-for-Collaborative-Robots/tree/main}{Developed a ROS-based interactive game for Baxter robot using YOLOv6 and DeepSORT for object tracking.}
    }
    
\experience
    {Autonomous Nerf Blaster Mobile Robot}
    {}
    {}
    {November 2021 – December 2021}
    {
        \item  Created a mobile robot with a modified Nerf blaster to autonomously track and fire at a moving shield.
    }


% Education 
\vspace{-0.5em}
\heading{Education}
\vspace{-1mm}%
\noindent\textbf{Rochester Institute of Technology}, Rochester, NY \\
Master of Science in Electrical Engineering \hfill Cumulative GPA: 3.92 \\
\textit{Specialization in Robotics and AI/ML} \\
\noindent\textbf{Rochester Institute of Technology}, Rochester, NY \\
Bachelor of Science in Electrical Engineering, \textit{Summa Cum Laude} \hfill Cumulative GPA: 3.86 \\
% \textbf{Applicable Courses:} Robotic-Systems, Principles-of-Robotics, Advanced-Robotics, Intro-to-Artificial-Intelligence, AI-Explorations, Biorobotics/Machine-Learning, Deep-Learning, Robot-Perception




\end{document}
